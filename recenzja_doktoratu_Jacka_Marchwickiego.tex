%%%%%%%%%%%%%%%%%%%%%%%%%%%%%%%%%%%%%%%%%%%%%%%%%%%%%%%%%%%%%%%%%%%%%%%%%
%                                                                       %
%            Recenzja doktoratu Jacka Marchwickiego                     %
%                                                                       %
%%%%%%%%%%%%%%%%%%%%%%%%%%%%%%%%%%%%%%%%%%%%%%%%%%%%%%%%%%%%%%%%%%%%%%%%%


\newcommand{\real}{{\mathbb R}}
\newcommand{\dideal}{d^0(\mathsf{V(X)})}
%%%%%%%%%%%%%%%%%%%%%% Calligraphic font commands %%%%%%%%%%%%%%%%%%%%%%%%%%
\newcommand{\cA}{{\cal A}}
\newcommand{\cB}{{\cal B}}
\newcommand{\cC}{{\cal C}}
\newcommand{\cD}{{\cal D}}
\newcommand{\cE}{{\cal E}}
\newcommand{\cF}{{\cal F}}
\newcommand{\cG}{{\cal G}}
\newcommand{\cH}{{\cal H}}
\newcommand{\cI}{{\cal I}}
\newcommand{\cJ}{{\cal J}}
\newcommand{\cK}{{\cal K}}
\newcommand{\cL}{{\cal L}}
\newcommand{\cM}{{\cal M}}
\newcommand{\cN}{{\cal N}}
\newcommand{\cO}{{\cal O}}
\newcommand{\cP}{{\cal P}}
\newcommand{\cQ}{{\cal Q}}
\newcommand{\cR}{{\cal R}}
\newcommand{\cS}{{\cal S}}
\newcommand{\cT}{{\cal T}}
\newcommand{\cU}{{\cal U}}
\newcommand{\cV}{{\cal V}}
\newcommand{\cW}{{\cal W}}
\newcommand{\cX}{{\cal X}}
\newcommand{\cY}{{\cal Y}}
\newcommand{\cZ}{{\cal Z}}

\newcommand{\continuum}{\mathfrak{c}}
\newcommand{\dummy}{{\tt UZUPELNIC!!!}}

\documentclass[12pt]{article}
\usepackage{amsfonts}
\usepackage{amsthm} 

%%%%%%%%%%%%%%% Polish letter packages %%%%%%%%%%%%%%%%%%%%%%%%%%%%%%%%
\usepackage[polish]{babel}
%\usepackage[latin2]{inputenc}
\usepackage[utf8]{inputenc}
\usepackage{t1enc}


%%%To print the date and time on each page
%%% comment out if not needed (next 14 lines)
\makeatletter
{\newcount\@hour}
{\newcount\@minute}
\def\timenow{\@hour=\time \divide\@hour by 60
\number\@hour:
  \multiply\@hour by 60 \@minute=\time
  \global\advance\@minute by -\@hour
  \ifnum\@minute<10 0\number\@minute\else
  \number\@minute\fi}
\def\ctimenow{\hfil{\tt \jobname.tex, \today~Time: \timenow }\hfil}
      \let\@oddfoot\ctimenow\let\@evenfoot\ctimenow
\makeatother

\pagestyle{plain}
%\markboth{{\bf Andrzej Nowik}}{}
 
\begin{document}
\large

\begin{flushleft}
Dr. hab Andrzej Nowik, prof UG \hfill Gdańsk, 9.04.2018 \\
Uniwersytet Gdański \\
Instytut Matematyki \\
Wita Stwosza 57 \\
80 -- 952 Gdańsk \\
e-mail: andrzej@mat.ug.edu.pl\\
\end{flushleft}
\vspace{3mm}
\begin{center}{\bf \LARGE
Recenzja rozprawy doktorskiej }
\end{center}
\begin{center}
{\LARGE \bf
Magistra Jacka Marchwickiego
}
\end{center}
\begin{center}
\bf \Large pt.
\end{center}
\begin{center} \LARGE
\bf Selected problems of series in finite
\end{center}

\begin{center} \LARGE
\bf dimensional spaces
\end{center}

%\begin{center}
%Napisanej pod kierunkiem dr hab. Szymona Głąba
%\end{center}

  Oddana do recenzji dysertacja doktorska pana Jacka Marchwickiego 
napisana jest w języku angielskim, liczy $92$
strony na których Autor zawarł kolejno: wprowadzenie, trzy
rozdzia\-ły na które została podzielona zasadnicza część rozprawy 
oraz obszerną biblio\-grafię zawie\-ra\-ją\-cą
$49$ pozycji. Promotorem Doktoranta jest dr hab. Szymon Głąb
z Politechniki Łódzkiej. Każdy rozdział omawianej rozprawy 
zaczyna się od krótkiego zarysu tematyki w nim poruszanej
a kończy się podsekcją z pytaniami otwartymi.

  Zdecydowana większość wyników zaprezentowanych w rozprawie
została opublikowana w trzech pracach autora (wśród których
jedna jest samodzielna, pozostałe dwie wspólne z Promotorem).

  Kluczowymi pojęciami w recenzowanej dysertacji są
rozmaite zbiory ,,stowarzyszone'' z zadanym szeregiem
elementów, przy czym rodzaj szeregu nie jest w pracy ograniczony,
to znaczy mogą to być szeregi liczbowe, szeregi wektorowe w 
przestrzeni euklidesowej a nawet szeregi w przestrzeni Banacha.
  Mianowicie, dla każdego szeregu $\sum_{n=1}^\infty x_n$
rozważamy zbiory:
$A(x_n)$ - zbiór wszystkich możliwych sum jakie są
do osiągnięcia przez wybór ,,podszeregu'' - tak zwany
''achievement set'';

$SR(x_n)$ - zbiór wszystkich możliwych sum jakie są
do osiągnięcia przez dokonanie permutacji wyrazów
danego szeregu - oznaczenie to akronim od słów 
''sum range'';

$LIM(\sum_{n=1}^\infty x_n)$ - zbiór wszystkich punktów
skupienia ciągu sum częściowych szeregu.

  Oprócz tego w rozprawie przejawiają się też rozmaite ,,mutacje''
tych klas, jak choćby wersje ideałowe tych rodzin.
  
  Niemal cała dysertacja doktorska zawiera analizę
własności tych zbiorów oraz relacji między tak zdefiniowanymi
klasami.
  

  Przystąpię teraz do omówienia treści wspomnianych trzech rozdziałów rozprawy:

\section{Omówienie poszczególnych rozdziałów}

\section{Uwagi ogólne}

%%%\section{Uwagi krytyczne}
%%%%Pisząc o mankamentach, chciałbym tylko wspomnieć iż (...) 

\section{Lista literówek oraz uwag do pracy}
\section{Konkluzja}
\end{document}

