%%%%%%%%%%%%%%%%%%%%%%%%%%%%%%%%%%%%%%%%%%%%%%%%%%%%%%%%%%%%%%%%%%%%%%%%%
%                                                                       %
%            Recenzja doktoratu Jacka Marchwickiego                     %
%                                                                       %
%%%%%%%%%%%%%%%%%%%%%%%%%%%%%%%%%%%%%%%%%%%%%%%%%%%%%%%%%%%%%%%%%%%%%%%%%


\newcommand{\real}{{\mathbb R}}
\newcommand{\dideal}{d^0(\mathsf{V(X)})}
%%%%%%%%%%%%%%%%%%%%%% Calligraphic font commands %%%%%%%%%%%%%%%%%%%%%%%%%%
\newcommand{\cA}{{\cal A}}
\newcommand{\cB}{{\cal B}}
\newcommand{\cC}{{\cal C}}
\newcommand{\cD}{{\cal D}}
\newcommand{\cE}{{\cal E}}
\newcommand{\cF}{{\cal F}}
\newcommand{\cG}{{\cal G}}
\newcommand{\cH}{{\cal H}}
\newcommand{\cI}{{\cal I}}
\newcommand{\cJ}{{\cal J}}
\newcommand{\cK}{{\cal K}}
\newcommand{\cL}{{\cal L}}
\newcommand{\cM}{{\cal M}}
\newcommand{\cN}{{\cal N}}
\newcommand{\cO}{{\cal O}}
\newcommand{\cP}{{\cal P}}
\newcommand{\cQ}{{\cal Q}}
\newcommand{\cR}{{\cal R}}
\newcommand{\cS}{{\cal S}}
\newcommand{\cT}{{\cal T}}
\newcommand{\cU}{{\cal U}}
\newcommand{\cV}{{\cal V}}
\newcommand{\cW}{{\cal W}}
\newcommand{\cX}{{\cal X}}
\newcommand{\cY}{{\cal Y}}
\newcommand{\cZ}{{\cal Z}}

\newcommand{\continuum}{\mathfrak{c}}
\newcommand{\dummy}{{\tt UZUPELNIC!!!}}

\documentclass[12pt]{article}
\usepackage{amsfonts}
\usepackage{amsthm} 

%%%%%%%%%%%%%%% Polish letter packages %%%%%%%%%%%%%%%%%%%%%%%%%%%%%%%%
\usepackage[polish]{babel}
%\usepackage[latin2]{inputenc}
\usepackage[utf8]{inputenc}
\usepackage{t1enc}


%%%To print the date and time on each page
%%% comment out if not needed (next 14 lines)
\makeatletter
{\newcount\@hour}
{\newcount\@minute}
\def\timenow{\@hour=\time \divide\@hour by 60
\number\@hour:
  \multiply\@hour by 60 \@minute=\time
  \global\advance\@minute by -\@hour
  \ifnum\@minute<10 0\number\@minute\else
  \number\@minute\fi}
\def\ctimenow{\hfil{\tt \jobname.tex, \today~Time: \timenow }\hfil}
      \let\@oddfoot\ctimenow\let\@evenfoot\ctimenow
\makeatother

\pagestyle{plain}
%\markboth{{\bf Andrzej Nowik}}{}
 
\begin{document}
%\large - w komentarz aby użyć jednak standardowej czcionki bowiem 
%%%% ta duża chyba wygląda zbyt pretensjonalnie

\begin{flushleft}
Dr. hab Andrzej Nowik, prof UG \hfill Gdańsk, 23.04.2018 \\
Uniwersytet Gdański \\
Instytut Matematyki \\
Wita Stwosza 57 \\
80 -- 952 Gdańsk \\
e-mail: andrzej@mat.ug.edu.pl\\
\end{flushleft}
\vspace{3mm}
\begin{center}{\bf \LARGE
Recenzja rozprawy doktorskiej }
\end{center}
\begin{center}
{\LARGE \bf
Magistra Jacka Marchwickiego
}
\end{center}
\begin{center}
\bf \Large pt.
\end{center}
\begin{center} \LARGE
\bf Selected problems of series in finite
\end{center}

\begin{center} \LARGE
\bf dimensional spaces
\end{center}

%\begin{center}
%Napisanej pod kierunkiem dr hab. Szymona Głąba, prof PŁ
%\end{center}

  Oddana do recenzji dysertacja doktorska pana Jacka Marchwickiego 
napisana jest w języku angielskim, liczy $92$
strony na których Autor zawarł kolejno: wprowadzenie, trzy
rozdzia\-ły na które została podzielona zasadnicza część rozprawy 
oraz obszerną biblio\-grafię zawie\-ra\-ją\-cą
$49$ pozycji. Promotorem Doktoranta jest dr hab.inż. Szymon Głąb,
profesor nadzwyczajny Politechniki Łódzkiej. Każdy rozdział omawianej rozprawy 
zaczyna się od krótkiego zarysu tematyki w nim poruszanej
a kończy się podsekcją z pytaniami otwartymi.

  Zdecydowana większość wyników zaprezentowanych w rozprawie
została opublikowana w trzech pracach autora (wśród których
jedna jest samodzielna, pozostałe dwie wspólne z Promotorem).

  Kluczowymi pojęciami recenzowanej dysertacji są
rozmaite zbiory ,,stowarzyszone'' z zadanym szeregiem
elementów (przy czym typ elementów szeregu nie jest w pracy ograniczony,
to znaczy mogą to być szeregi liczbowe, szeregi wektorowe w 
przestrzeni euklidesowej a nawet szeregi w przestrzeni Banacha).
  Otóż, dla każdego szeregu $\sum_{n=1}^\infty x_n$
rozważamy zbiory:
$A(x_n)$ - zbiór wszystkich możliwych sum jakie są
do osiągnięcia przez wybór ,,podszeregu'' - tak zwany
''achievement set'';

$SR(x_n)$ - zbiór wszystkich możliwych sum jakie są
do osiągnięcia przez dokonanie permutacji wyrazów
danego szeregu (oznaczenie jest akronimem słów 
''sum range'');

$LIM(\sum_{n=1}^\infty x_n)$ - zbiór wszystkich punktów
skupienia ciągu sum częściowych szeregu. Oprócz tego w rozprawie 
przejawiają się też rozmaite ,,mutacje''
tych klas, jak choćby wersje ideałowe tych rodzin. 
Niemal cała dysertacja doktorska zawiera analizę
własności tych zbiorów oraz relacji między tak zdefiniowanymi
klasami. Przystąpię teraz do omówienia treści wspomnianych trzech rozdziałów rozprawy:

\section{Omówienie poszczególnych rozdziałów}

  Obszerny rozdział pierwszy rozpada się na dwie niezależne części.
Pierwsza część zawiera różnorodne rezultaty dotyczące ogólnego
pytania o charakteryzację zbiorów jakie mogą być postaci $A(v_n)$
oraz o relacje między zbiorami $A(v_n)$ a $SR(v_n)$, przy czym
rozważane są szeregi wektorów na płaszczyźnie.
Okazuje się że zbiór $A(v_n)$ dla szeregu zbieżnego warunkowo 
może być: wykresem funkcji,
zbiorem brzegowym i jednocześnie gęstym, nie musi być zbiorem
$F_\sigma$ i $G_\delta$, ale może być jednak po prostu zbiorem
otwartym różnym od całej płaszczyzny. Aby to wykazać Autor rozprawy 
wprawnie konstruuje odpowiednie przykłady szeregów. Moim zdaniem niezwykle
intrygującym wynikiem z omawianego podrozdziału jest twierdzenie
1.11 mówiące o tym że przy pewnych założeniach odnośnie szeregu
wektorowego jego zbiór $A(v_n)$ jest zbiorem borelowskim 
%należącym do klasy w hierarchii borelowskiej o indeksie co najmniej dwa -
%czyli po prostu 
nie będącym wszakże zbiorem typu $F_\sigma$ a także 
$G_\delta$. Natychmiast po rzeczonym twierdzeniu pojawia się uzupełniający
je przykład (przykład 1.12) w którym zbiór $A(v_n)$ jest wykresem funkcji 
który nie jest zbiorem $F_\sigma$ oraz zbiorem $G_\delta$. 

  Rozdział pierwszy nabiera rumieńców w swojej drugiej części, 
która poświęcona jest zbadaniu warunków jakie powinien spełniać
warunkowo zbieżny szereg wektorów by jego zbiór $A(v_n)$
był całą płaszczyzną - tutaj Autor posiłkuje się pojęciem tak 
zwanych wektorów L\'evy szeregu i w tym języku formułuje 
odpowiedni warunek. Moim zdaniem jest to chyba jedna z najciekawszych, 
choć jednocześnie też najtrudniejsza do przeczytania 
z uwagi na wyrafinowanie techniczne dowodów, część rozprawy.
  
  Cały rozdział drugi poświęcony został  
%%% usunąłem tu przymiotnik ,,obszerny'' bo rozdział drugi bynajmniej nie jest taki
%%% obszerny, przynajmniej w porównaniu do rozdziału pierwszego
tak zwanym ideałowym wersjom zbiorów $A(x_n)$ oraz
$SR(x_n)$. Rozmaite próby uogólnienia klasycznych twierdzeń
i pojęć na przypadek ideałowy przykuwają współcześnie
uwagę wielu matematyków pracujących na styku teorii zbiorów 
i analizy matematycznej. Dbałość o zwięzłą postać niniejszej
recenzji nie pozwala mi na dłuższe omówienie rozdziału drugiego,
zważywszy na ogrom zaprezentowanych w nim wyników. Poprzestanę zatem
jedynie na zwróceniu uwagi na sympatyczne twierdzenie 2.14 które 
jest jak gdyby naturalnym dopełnieniem badań zapoczątkowanych
przez Rafała Filipówa i Piotra Szucę a dotyczących 
postaci zbioru $SR_I(x_n)$ dla warunkowo zbieżnego szeregu.
Podczas gdy zacytowani uprzednio matematycy scharakteryzowali
ideały dla których ów zbiór jest całą prostą, to
Autor niejako zamykając w pełni sprawę wyjaśnia możliwą
postać interesującego nas zbioru.
  
  W rozdziale trzecim Autor dogłębnie rozstrząsa 
trzeci ze wspomnianych zbiorów a mianowicie
zbiór $LIM(\sum_{n=1}^\infty v_n)$. Lwia część 
tego rozdziału stanowi rozwinięcie a nawet
skorygowanie dorobku z pracy Victora Klee
(cytowanej w bibliografii jako pozycja \big[28\big]).
Gwoździem programu jest niewątpliwie przykład
3.3 który stanowi kontrprzykład na rezultat
z cytowanej pracy Victora Klee. Przykład ten
jest niezwykle misternie skonstruowany a
sam pomysł zadziwia swą elegancją. Na podkreślenie zasługuje 
ogromna erudycja Doktoranta który nie tylko
szczegółowo przeanalizował cytowaną
pracę (nadmienię że nie ułatwia bynajmniej jej lektury fakt
iż posługuje się ona odmiennymi od 
używanych we współczesnych pracach oznaczeniami)
ale nawet wypunktował explicite miejsce gdzie autor 
popełnił błąd.

\section{Uwagi ogólne}

  Struktura recenzowanej dysertacji doktorskiej jest
klarowna i odpowiednio wyważona. Techniki dowodowe 
są bardzo pomysłowe - dla przykładu uwagę recenzenta 
przykuło szczególnie wykorzystanie narzędzia jakim jest
pojęcie liczb Liouville'a w dowodzie że szereg z przykładu
1.15 ma gęsty lecz o mierze zero zbiór $A(v_n)$, a także
zręczne użycie wektorów L\'evy, pojęcia nieco już 
zapomnianego, ale na powrót 
przeżywającego (także dzięki dorobkowi Doktoranta i jego Promotora) 
swój renesans. 
  Dowody są prawidłowe, nie znalazłem w nich luk czy 
błędów, choć lektura niniejszej pracy stawia przez
czytelnikiem bardzo wysokie wymagania - styl rozprawy
jest bardziej zbliżony do stylu artykułu w fachowym czasopiśmie
niż do zwyczajowego stylu pracy doktorskiej - niewykluczone
iż czytelnikowi pomogłoby czasami dodanie nieformalnych
wyjasnień które pozwoliłyby szybciej zrozumieć niektóre 
rozumowania. Z drugiej strony mimo zwartości treści 
rozprawy i tak przytłacza ona objętością swych niemalże
$100$ stron, więc sugerowana przeze mnie lakoniczność
niektórych rozumowań może być przyjęta ze zrozumieniem.
%%%  Na podkreślenie zasługuje bardzo zręczny 
%%%% - tu chcę napisać że bardzo mi się
%%%% podoba stylistyka autora polegająca na kończeniu
%%%% każdego rodziału listą pytań.

%%%\section{Uwagi krytyczne}
%%%%Pisząc o mankamentach, chciałbym tylko wspomnieć iż (...) 

\section{Strona formalna pracy}

  Jak już wspomniałem, rozprawa została napisana w języku angielskim. 
Autor niniejszej recenzji nie czuje się
bynajmniej specjalistą w dziedzinie poprawności lingwistycznej, jednak
mam odczucie że liczba błędów stylistycznych i językowych jest 
nieomalże znikoma. Jednak, jak powszechnie wiadomo, w pracy mającej ponad dziewięćdziesiąt
stron nie sposób ustrzec się małych błędów typograficznych. Mimo
wszystko wytropiłem ich zadziwiająco mało i są one 
nieistotne dla klarowności dysertacji. Aby nie być 
jednakże gołosłownym, załączam kilka z nich:

\begin{itemize}
\item
  $3^4$: ($c_{00}$)
\item 
  $15^{11}$ osobliwa zbitka postaci: ''with onSR(T''
\item
  $27^5$ literówka: ''subsequece''
%%% waham się nie wiem czy o tym napomknąć
%\item 
%  w wielu miejscach pracy - brak ostrego akcentu w nazwisku ''L\'evy''
\end{itemize}

\section{Konkluzja}
Podsumowując, recenzowana dysertacja doktorska stoi
na bardzo dobrym poziomie naukowym. Jej tematyka jest
zdecydowanie nowoczesna i idealnie wtapia się w najnowsze
trendy badań z pograniczna analizy matematycznej i teorii mnogości.
Zamieszczone na końcu każdego rozdziału intrygujące otwarte problemy
niechybnie mogą posłużyć za swego rodzaju ,,drogowskazy'' 
do kontynuacji dalszych badań.
  Zważywszy więc na wszystkie wyłuszczone powyżej argumenty,
biorąc pod uwagę ogrom dorobku naukowego Doktoranta 
% usuwam, bo to sformułowanie pojawiło się już wcześniej
%(dbałość o szczupłość recenzji pozwoliła mi na omówienie tylko wybranych
%kilku wyników z recenzowanej rozprawy) 
stwierdzam iż omawiana 
praca z nadmiarem spełnia zarówno zwyczajowe jak i ustawowe
wymagania stawiane rozprawom doktorskim. W związku z tym
{\bf wnoszę o dopuszczenie} Magistra Jacka Marchwickiego do dalszych
etapów przewodu doktorskiego.
  Ponieważ zaś w niniejszej rozprawie został wytropiony i 
skorygowany błąd z klasycznej, napisanej ponad pół wieku temu przez znanego
autora pracy, chciałbym zgłosić Radzie Wydziału Fizyki 
Technicznej, Informatyki i Matematyki Stosowanej Politechniki
Łódzkiej wniosek o {\bf wyróżnienie} niniejszej pracy doktorskiej.


\end{document}

